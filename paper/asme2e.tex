%%%%%%%%%%%%%%%%%%%%%%%%%%% asme2e.tex %%%%%%%%%%%%%%%%%%%%%%%%%%%%%%%
% Template for producing ASME-format articles using LaTeX            %
% Written by   Harry H. Cheng                                        %
%              Integration Engineering Laboratory                    %
%              Department of Mechanical and Aeronautical Engineering %
%              University of California                              %
%              Davis, CA 95616                                       %
%              Tel: (530) 752-5020 (office)                          %
%                   (530) 752-1028 (lab)                             %
%              Fax: (530) 752-4158                                   %
%              Email: hhcheng@ucdavis.edu                            %
%              WWW:   http://iel.ucdavis.edu/people/cheng.html       %
%              May 7, 1994                                           %
% Modified: February 16, 2001 by Harry H. Cheng                      %
% Modified: January  01, 2003 by Geoffrey R. Shiflett                %
% Use at your own risk, send complaints to /dev/null                 %
%%%%%%%%%%%%%%%%%%%%%%%%%%%%%%%%%%%%%%%%%%%%%%%%%%%%%%%%%%%%%%%%%%%%%%

%%% use twocolumn and 10pt options with the asme2e format
\documentclass[twocolumn,10pt]{asme2e}
\special{papersize=8.5in,11in}
%% The class has several options
%  onecolumn/twocolumn - format for one or two columns per page
%  10pt/11pt/12pt - use 10, 11, or 12 point font
%  oneside/twoside - format for oneside/twosided printing
%  final/draft - format for final/draft copy
%  cleanfoot - take out copyright info in footer leave page number
%  cleanhead - take out the conference banner on the title page
%  titlepage/notitlepage - put in titlepage or leave out titlepage
%  
%% The default is oneside, onecolumn, 10pt, final

%%% Replace here with information related to your conference
\confshortname{IDETC/CIE 2018}
\conffullname{the ASME 2018 International Design Engineering Technical Conferences \&
Computers in Engineering Conferences}

%%%%% for date in a single month, use
\confdate{26-29}
\confmonth{August}
\confyear{2018}
\confcity{Quebec City, Quebec}
\confcountry{Canada}

%%% Replace DETC2009/MESA-12345 with the number supplied to you 
%%% by ASME for your paper.
\papernum{DETC2018-DRAFT}

%%% You need to remove 'DRAFT: ' in the title for the final submitted version.
\title{DRAFT: DESIGNING IMPROVED TEAMS FOR DATA SCIENCE}

%%% first author
\author{Christopher McComb
    \affiliation{
	School of Engineering Design, Technology, \\and Professional Programs\\
	The Pennsylvania State University\\
	University Park, PA 16802\\
    Email: mccomb@psu.edu
    }	
}

%%% second author
\author{First Coauthor\thanks{Address all correspondence to this author.} \\
       {\tensfb Second Coauthor}
    \affiliation{Department or Division Name\\
	Company or College Name\\
	City, State (spelled out), Zip Code\\
	Country (only if not U.S.)\\
	Email address (if available)
    }
}

\begin{document}

\maketitle    

%%%%%%%%%%%%%%%%%%%%%%%%%%%%%%%%%%%%%%%%%%%%%%%%%%%%%%%%%%%%%%%%%%%%%%
\begin{abstract}
Teams are ubiquitous
Often, we assume that teams are better at solving problems that individuals working independently.
Although early work on software development helped to develop team and organizational theory, there has been little analysis of teams in modern distributed data science competitions.
This paper analyzes data science competition teams by considering them to be design teams creating a product. Grounding the analysis by relating it to design is an important decision.
Recent work in engineering, design, and psychology has indicated that teams may not be the problem-solving panacea that they were once thought to be. In fact.


\end{abstract}

%%%%%%%%%%%%%%%%%%%%%%%%%%%%%%%%%%%%%%%%%%%%%%%%%%%%%%%%%%%%%%%%%%%%%%
\section{INTRODUCTION}
Teams

Significant disagreement exists over whether

The Mythical man-month

\section{DATA}
This paper focuses on the
This work specifically utilizes the Meta Kaggle dataset \cite{metakaggle}.


\section{ANALYSIS}

\subsection{Metrics}
\subsubsection{Solution Quality}
\subsubsection{Effort}

\subsection{Team Types}
\subsubsection{Individuals}
\subsubsection{True Teams}
\subsubsection{Nominal Teams}


\section{RESULTS}

\section{CONCLUSIONS}

\begin{acknowledgment}
This material is based upon work supported by the Defense Advanced Research Projects Agency through cooperative agreement N66001-17-1-4064. Any opinions, findings, and conclusions or recommendations expressed in this paper are those of the authors and do not necessarily reflect the views of the sponsors.
\end{acknowledgment}

\section*{MATHEMATICS}

Equations should be numbered consecutively beginning with (1) to the end of the paper, including any appendices.  The number should be enclosed in parentheses and set flush right in the column on the same line as the equation.  An extra line of space should be left above and below a displayed equation or formula. \LaTeX\ can automatically keep track of equation numbers in the paper and format almost any equation imaginable. An example is shown in Eqn.~(\ref{eq_ASME}). The number of a referenced equation in the text should be preceded by Eqn.\ unless the reference starts a sentence in which case Eqn.\ should be expanded to Equation.

\begin{equation}
f(t) = \int_{0_+}^t F(t) dt + \frac{d g(t)}{d t}
\label{eq_ASME}
\end{equation}

%%%%%%%%%%%%%%%%%%%%%%%%%%%%%%%%%%%%%%%%%%%%%%%%%%%%%%%%%%%%%%%%%%%%%%
\section*{FIGURES AND TABLES}

All figures should be positioned at the top of the page where possible.  All figures should be numbered consecutively and captioned; the caption uses all capital letters, and centered under the figure as shown in Fig.~\ref{figure_ASME}. All text within the figure should be no smaller than 7~pt. There should be a minimum two line spaces between figures and text. The number of a referenced figure or table in the text should be preceded by Fig.\ or Tab.\ respectively unless the reference starts a sentence in which case Fig.\ or Tab.\ should be expanded to Figure or Table.


%%%%%%%%%%%%%%%%%%%%%%%%%%%%%%%%%%%%%%%%%%%%%%%%%%%%%%%%%%%%%%%%%%%%%%
%%%%%%%%%%%%%%%% begin figure %%%%%%%%%%%%%%%%%%%
\begin{figure}[t]
\begin{center}
\setlength{\unitlength}{0.012500in}%
\begin{picture}(115,35)(255,545)
\thicklines
\put(255,545){\framebox(115,35){}}
\put(275,560){Beautiful Figure}
\end{picture}
\end{center}
\caption{THE FIGURE CAPTION USES CAPITAL LETTERS.}
\label{figure_ASME} 
\end{figure}
%%%%%%%%%%%%%%%% end figure %%%%%%%%%%%%%%%%%%% 
%%%%%%%%%%%%%%%%%%%%%%%%%%%%%%%%%%%%%%%%%%%%%%%%%%%%%%%%%%%%%%%%%%%%%%


%%%%%%%%%%%%%%%%%%%%%%%%%%%%%%%%%%%%%%%%%%%%%%%%%%%%%%%%%%%%%%%%%%%%%%
%%%%%%%%%%%%%%% begin table   %%%%%%%%%%%%%%%%%%%%%%%%%%
\begin{table}[t]
\caption{THE TABLE CAPTION USES CAPITAL LETTERS, TOO.}
\begin{center}
\label{table_ASME}
\begin{tabular}{c l l}
& & \\ % put some space after the caption
\hline
Example & Time & Cost \\
\hline
1 & 12.5 & \$1,000 \\
2 & 24 & \$2,000 \\
\hline
\end{tabular}
\end{center}
\end{table}
%%%%%%%%%%%%%%%% end table %%%%%%%%%%%%%%%%%%% 
%%%%%%%%%%%%%%%%%%%%%%%%%%%%%%%%%%%%%%%%%%%%%%%%%%%%%%%%%%%%%%%%%%%%%%

All tables should be numbered consecutively and  captioned; the caption should use all capital letters, and centered above the table as shown in Table~\ref{table_ASME}. The body of the table should be no smaller than 7 pt.  There should be a minimum two line spaces between tables and text.


%%%%%%%%%%%%%%%%%%%%%%%%%%%%%%%%
\bibliographystyle{asmems4}
\bibliography{asme2e}

\end{document}